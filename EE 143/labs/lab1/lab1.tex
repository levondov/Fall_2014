%%%%%%%%%%%%%%%%%%%%%%%%%%%%%%%%%%%%%%%%%
% University/School Laboratory Report
% LaTeX Template
% Version 3.1 (25/3/14)
%
% This template has been downloaded from:
% http://www.LaTeXTemplates.com
%
% Original author:
% Linux and Unix Users Group at Virginia Tech Wiki 
% (https://vtluug.org/wiki/Example_LaTeX_chem_lab_report)
%
% License:
% CC BY-NC-SA 3.0 (http://creativecommons.org/licenses/by-nc-sa/3.0/)
%
%%%%%%%%%%%%%%%%%%%%%%%%%%%%%%%%%%%%%%%%%

%----------------------------------------------------------------------------------------
%	PACKAGES AND DOCUMENT CONFIGURATIONS
%----------------------------------------------------------------------------------------

\documentclass{article}

\usepackage[version=3]{mhchem} % Package for chemical equation typesetting
\usepackage{siunitx} % Provides the \SI{}{} and \si{} command for typesetting SI units
\usepackage{graphicx} % Required for the inclusion of images
\usepackage{natbib} % Required to change bibliography style to APA
\usepackage{amsmath} % Required for some math elements 
\usepackage[utf8]{inputenc}
\usepackage{tikz,pgfplots}
\usepackage[letterpaper, margin=0.5in]{geometry}
\usepackage{float}
\usepackage{enumitem}
\usepackage{gensymb}
\usepackage[hidelinks]{hyperref}
\usepackage[all]{hypcap}
\usepackage{subfloat}
\usepackage{color}

% Roman numerials
\pagenumbering{arabic}

\setlength\parindent{0pt} % Removes all indentation from paragraphs

%\renewcommand{\labelenumi}{\alph{enumi}.} % Make numbering in the enumerate environment by letter rather than number (e.g. section 6)

%\usepackage{times} % Uncomment to use the Times New Roman font

% for some tables
\newcommand{\specialcell}[2][c]{%
  \begin{tabular}[#1]{@{}c@{}}#2\end{tabular}}
  
\newcommand{\me}{\mathrm{e}}
\providecommand{\e}[1]{\ensuremath{\times 10^{#1}}}

%%%%%%%%%%%% COLOR DEFINITIONS %%%%%%%%%%%%%

\definecolor{silicon}{RGB}{255,102,102}
\definecolor{oxide}{RGB}{145,150,110}
\definecolor{ioxide}{RGB}{175,180,135}
\definecolor{goxide}{RGB}{195,200,150}
\definecolor{poly}{RGB}{155,20,155}
\definecolor{spinglass}{RGB}{200,205,100}
\definecolor{n+}{RGB}{250,15,15}
\definecolor{aluminum}{RGB}{30,30,30}

%----------------------------------------------------------------------------------------
%	DOCUMENT INFORMATION
%----------------------------------------------------------------------------------------

%\title{Determination of the Atomic \\ Weight of Magnesium \\ CHEM 101} % Title

%\author{John \textsc{Smith}} % Author name

%\date{\today} % Date for the report

\begin{document}

%\maketitle % Insert the title, author and date

% If you wish to include an abstract, uncomment the lines below
% \begin{abstract}
% Abstract text
% \end{abstract}

%----------------------------------------------------------------------------------------
%	SECTION 1
%----------------------------------------------------------------------------------------

\section{Profiles \& Layout}
\subsection{}
\subsection{}
\subsection{}
\section{Process Procedures}
\subsection{}
\subsection{}
\subsection{}
\section{Calculations}
a) Film Thickness
\begin{figure}[H]
\centering
\begin{tabular}{c | c | c | c | c | c | c |}
Layer & \specialcell{Theoretical \\ calculation \\ (nm)} & \specialcell{Experimental \\ (nm)} & \% Error & \specialcell{Linewidths \\ (photoresist) \\ (nm)} & \specialcell{Linewidths \\ (after PR Strip) \\ (nm)} & \% Overetch \\ \hline

Field Oxide & 505.8 & 477.2 & 5.65 & ? & 3000 & ? \\ \hline
Polysilicon & ? & ? & ? & ? & ? & ? \\ \hline
Gate Oxide & 80.1 & 86.5 & 7.40 & 3628 & 4000 & ? \\ \hline
Intermed Oxide & 386.3 & 320 & 17.2 & ? & ? & ? \\ \hline
Aluminum & ? & ? & ? & 2088 & 2520 & ? \\ \hline

\end{tabular}
\end{figure}

b) Sheet Resistance
\section{Questions}
\begin{enumerate}
\item The lab uses positive photoresist. The lithography machine uses a light that has g line wavelength. The I and G lines refer to the wavelength of the light coming off the light bulb that reacts with photoresist. G line is roughly xx wavelength and I line is xx wavelength.
\end{enumerate}

\section{Appendix}
Film thickness calculation for oxide:
\begin{equation}
X_{ox} = \frac{0.5B}{B/A}\Big[\sqrt{1 + \frac{4}{B}(\frac{B}{A})^2(t + \tau)} - 1\Big]
\end{equation}
\begin{equation}
\tau = \frac{{X_i}^2}{B} + \frac{X_i}{B/A}
\end{equation}
where
\begin{align*}
B/A = D_o\me^{\frac{-E_a}{kT}} (\text{Use table 3.1 [1] to find }E_a \text{and } D_o) \\
B = D_o\me^{\frac{-E_a}{kT}} (\text{Use table 3.1 [1] to find }E_a \text{and } D_o) \\
t = \text{Time of oxide growth} \\ 
\tau = \text{Time of initial oxide growth already present} \\
X_i = \text{length of initial oxide growth}
\end{align*}
%----------------------------------------------------------------------------------------
%	SECTION 5
%----------------------------------------------------------------------------------------


%----------------------------------------------------------------------------------------
%	SECTION 6
%----------------------------------------------------------------------------------------

% Nothing right now

%----------------------------------------------------------------------------------------
%	BIBLIOGRAPHY
%----------------------------------------------------------------------------------------

\bibliographystyle{apalike}

\bibliography{sample}

%----------------------------------------------------------------------------------------


\end{document}

