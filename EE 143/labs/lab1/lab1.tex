%%%%%%%%%%%%%%%%%%%%%%%%%%%%%%%%%%%%%%%%%
% University/School Laboratory Report
% LaTeX Template
% Version 3.1 (25/3/14)
%
% This template has been downloaded from:
% http://www.LaTeXTemplates.com
%
% Original author:
% Linux and Unix Users Group at Virginia Tech Wiki 
% (https://vtluug.org/wiki/Example_LaTeX_chem_lab_report)
%
% License:
% CC BY-NC-SA 3.0 (http://creativecommons.org/licenses/by-nc-sa/3.0/)
%
%%%%%%%%%%%%%%%%%%%%%%%%%%%%%%%%%%%%%%%%%

%----------------------------------------------------------------------------------------
%	PACKAGES AND DOCUMENT CONFIGURATIONS
%----------------------------------------------------------------------------------------

\documentclass{article}

\usepackage[version=3]{mhchem} % Package for chemical equation typesetting
\usepackage{siunitx} % Provides the \SI{}{} and \si{} command for typesetting SI units
\usepackage{graphicx} % Required for the inclusion of images
\usepackage{natbib} % Required to change bibliography style to APA
\usepackage{amsmath} % Required for some math elements 
\usepackage[utf8]{inputenc}
\usepackage{tikz,pgfplots}
\usepackage[letterpaper, margin=0.5in]{geometry}
\usepackage{float}
\usepackage{enumitem}
\usepackage{gensymb}
\usepackage[hidelinks]{hyperref}
\usepackage[all]{hypcap}
\usepackage{subfloat}
\usepackage{color}

% Roman numerials
\pagenumbering{arabic}

\setlength\parindent{0pt} % Removes all indentation from paragraphs

%\renewcommand{\labelenumi}{\alph{enumi}.} % Make numbering in the enumerate environment by letter rather than number (e.g. section 6)

%\usepackage{times} % Uncomment to use the Times New Roman font

% for some tables
\newcommand{\specialcell}[2][c]{%
  \begin{tabular}[#1]{@{}c@{}}#2\end{tabular}}
  
\newcommand{\me}{\mathrm{e}}
\providecommand{\e}[1]{\ensuremath{\times 10^{#1}}}

\newcommand{\eqname}[1]{\tag*{#1}}

\DeclareMathOperator\erfc{erfc}

%%%%%%%%%%%% COLOR DEFINITIONS %%%%%%%%%%%%%

\definecolor{silicon}{RGB}{255,102,102}
\definecolor{oxide}{RGB}{145,150,110}
\definecolor{ioxide}{RGB}{175,180,135}
\definecolor{goxide}{RGB}{195,200,150}
\definecolor{poly}{RGB}{155,20,155}
\definecolor{spinglass}{RGB}{200,205,100}
\definecolor{n+}{RGB}{250,15,15}
\definecolor{aluminum}{RGB}{30,30,30}

%----------------------------------------------------------------------------------------
%	DOCUMENT INFORMATION
%----------------------------------------------------------------------------------------

%\title{Determination of the Atomic \\ Weight of Magnesium \\ CHEM 101} % Title

%\author{John \textsc{Smith}} % Author name

%\date{\today} % Date for the report

\begin{document}

%\maketitle % Insert the title, author and date

% If you wish to include an abstract, uncomment the lines below
% \begin{abstract}
% Abstract text
% \end{abstract}

%----------------------------------------------------------------------------------------
%	SECTION 1
%----------------------------------------------------------------------------------------

\section{Profiles \& Layout}
\subsection{}
\subsection{}
\subsection{}
\section{Process Procedures}
\subsection{}
\subsection{}
\subsection{}
\section{Calculations}
a) Film Thickness
\begin{figure}[H]
\centering
\begin{tabular}{c | c | c | c | c | c | c}
Layer & \specialcell{Theoretical \\ calculation \\ (nm)} & \specialcell{Experimental \\ (nm)} & \% Error & \specialcell{Linewidths \\ (photoresist) \\ (nm)} & \specialcell{Linewidths \\ (after PR Strip) \\ (nm)} & \% Overetch \\ \hline

Field Oxide & 505.8 & 477.2 & 5.65 & ? & 3000 & ? \\ \hline
Polysilicon & ? & ? & ? & ? & ? & ? \\ \hline
Gate Oxide & 80.1 & 86.5 & 7.40 & 3628 & 4000 & ? \\ \hline
Intermed Oxide & 386.3 & 320 & 17.2 & ? & ? & ? \\ \hline
Aluminum & ? & ? & ? & 2088 & 2520 & ? \\ \hline

\end{tabular}
\end{figure}

b) Sheet Resistance \\
c) Overetch

\begin{description}[style = nextline]
\item[1) Theoretical and experimental thicknesses of field oxide, gate and intermediate 
oxides (Include orientation dependence of oxidation rate but not impurity 
dependence) (9 points)]
For details on the theoretical oxide thickness calculations see Appendix \textcolor{blue}{\ref{sec:oxide}}.
%NOTE REDO DRY OXIDE AND INCLUDE TAO OF 25nm!
\begin{figure}[H]
\centering
\begin{tabular}{c | c | c | c}
Layer & \specialcell{Theoretical \\ (nm)} & \specialcell{Experimental \\ (nm)} & \% Error \\ \hline
Field Oxide & 505.8 & 477.2 & 5.65 \\ \hline
Gate Oxide & 80.1 & 86.5 & 7.40  \\ \hline
Intermed Oxide & 386.3 & 320 & 17.2 \\ \hline
\end{tabular}
\end{figure}

\item[2) Junction depths after pre-diffusion and drive-in (theoretical, assume only 
phosphorous doping with surface concentration limited by solid solubility). You 
must consider the effect of the initial ion implantation. For pre-deposition you may 
use the box approximation, but for drive-in you must use the half-gaussian 
calculation. Why is this? (10 points)]

For details on the junction depth calculations see Appendix \textcolor{blue}{\ref{sec:jdepth}}.

\begin{figure}[H]
\centering
\begin{tabular}{c | c}
Step & \specialcell{Junction Depth \\ (nm)} \\ \hline
Pre-diffusion & 365 \\ \hline
Drive-in & 1000 \\ \hline
\end{tabular}
\end{figure}

\item[3) test]
test

\item[4) test]
test

\item[5) test]
test

\item[6) List an estimate of the Young's modulus, Poisson ratio, and coefficient of thermalexpansion for Si$\text{O}_2$, poly-Si, and Al films as deposited. (You can find these in a table in many physics/ME textbooks, or in a web-based search.) (2 points) ]
See table below

\begin{figure}[H]
\centering
\begin{tabular}{c || c | c | c}
Material & \specialcell{Young's modulus \\ (GPa)} & \specialcell{coeff. of \\ thermal expansion \\ ($K^{-1}$)} & \specialcell{Poisson's ratio \\ (a.u.)} \\ \hline
Si$\text{O}_2$ [2]& 57(dry)-70(wet) & 5.6 $ \cdot$ 10$ ^{-7}$  & 0.17 \\ \hline
poly-Si [1]& 169 $\pm$ 6.15 & 4.6 $ \cdot$ 10$ ^{-6}$ & 0.22 $\pm$ .011 \\ \hline
Aluminum [3]& 70 & 22.2 $\cdot$ 10$ ^{-6}$ & 0.33 \\ \hline
\end{tabular}
\caption{Values were found from various journals/articles found online ([1],[2], and[3]) and many of these values are calculated for thin films of the material.}
\end{figure}


\end{description}
\section{Questions}
\begin{enumerate}
\item The lab uses positive photoresist. The lithography machine uses a light that has g line wavelength. The I and G lines refer to the wavelength of the light coming off the light bulb that reacts with photoresist. G line is roughly xx wavelength and I line is xx wavelength.
\end{enumerate}






\section{Appendix}

\subsection{Oxide Thickness calculations}
\label{sec:oxide}
Film thickness calculation for oxides:
\begin{equation}
X_{ox} = \frac{0.5B}{B/A}\Big[\sqrt{1 + \frac{4}{B}(\frac{B}{A})^2(t + \tau)} - 1\Big]
\end{equation}
\begin{equation}
\tau = \frac{{X_i}^2}{B} + \frac{X_i}{B/A}
\end{equation}
where
\begin{align*}
B/A = D_o\me^{\frac{-E_a}{kT}} (\text{Use table 3.1 [1] to find }E_a \text{and } D_o) \\
B = D_o\me^{\frac{-E_a}{kT}} (\text{Use table 3.1 [1] to find }E_a \text{and } D_o) \\
t = \text{Time of oxide growth} \\ 
\tau = \text{Time of initial oxide growth already present} \\
X_i = \text{length of initial oxide growth}
\end{align*}

Example: Calculated oxide thickness of Intermed Oxide: \\
Given: 5 min dry oxidation at $1050\degree$C and 12 + 25 min wet oxidation and annealing at $1050\degree$C, calculated oxide growth. \\

First we consider the 5 min dry oxidation. Using table 3.1 [1] for a $<100>$ Silicon, and using dry oxidation, we see that for the linear rate constant (B/A), $E_A = 2.00$ eV and $D_o = 3.71\e{6} \, \mu$m/hr. For the parabolic rate constant (B), $E_A = 1.23$ eV and $D_o = 772\, \mu$m/hr. \\ \\
Using an arrhenius equation where $k =$ Boltzmann's constant, and $T = $ temperature. 
\begin{align*}
B/A &= D_o\me^{\frac{-E_A}{kT}} = 3.71\e{6}\me^{\frac{-2.00*1.602\e{-19}}{1.38\e{-23}(1050\degree\text{C} + 273)}} = 0.0887 \mu\text{m/hr} \\
B &= D_o\me^{\frac{-E_A}{kT}} = 772\me^{\frac{-1.23*1.602\e{-19}}{1.38\e{-23}(1050\degree\text{C} + 273)}} = 0.0159 \,{\mu\text{m}}^2\text{/hr}
\end{align*}
Since there is no initial oxide here, $\tau = 0$,
\begin{align*}
X_{ox} = \frac{0.5B}{B/A}\Big[\sqrt{1 + \frac{4}{B}(\frac{B}{A})^2(t + \tau)} - 1\Big] = \frac{0.5 (0.0159)}{0.0887}\Big[\sqrt{1 + \frac{4}{0.0159}(0.0887)^2(\frac{5\text{min}}{60\text{min/hr}} + 0)} - 1\Big] \approx 7.11 \, \text{nm}
\end{align*}

Now after this dry oxidation, we have a 37 minute wet oxidation at $1050 \degree$C. Using table 3.1 [1] again, but this time using the constants that apply for wet oxidation,

\begin{align*}
{B/A}_{\text{wet}} &= D_o\me^{\frac{-E_A}{kT}} = 9.70\e{7}\me^{\frac{-2.05*1.602\e{-19}}{1.38\e{-23}(1050\degree\text{C} + 273)}} = 1.50 \mu\text{m/hr} \\
B_{\text{wet}} &= D_o\me^{\frac{-E_A}{kT}} = 386\me^{\frac{-0.78*1.602\e{-19}}{1.38\e{-23}(1050\degree\text{C} + 273)}} = 0.411 \,{\mu\text{m}}^2\text{/hr}
\end{align*}

This time we do have an initial oxidation time $\tau$ because of the dry oxidation we did. Here $X_i$ is the oxide length we calculated for the dry oxidation growth,

\begin{align*}
\tau = \frac{{X_i}^2}{B_{\text{wet}}} + \frac{X_i}{{B/A}_{\text{wet}}} = \frac{0.00711^2}{0.411} + \frac{0.00711}{1.50} \approx 0.00488 \text{hrs}
\end{align*}

And finally our oxide growth is,
\begin{align*}
X_{ox} = \frac{0.5B}{B/A}\Big[\sqrt{1 + \frac{4}{B}(\frac{B}{A})^2(t + \tau)} - 1\Big] = \frac{0.5 (0.411)}{1.50}\Big[\sqrt{1 + \frac{4}{0.411}(1.50)^2(\frac{(12 + 25)\text{min}}{60\text{min/hr}} + 0.00488 \text{hrs})} - 1\Big] \approx 386.3 \, \text{nm}
\end{align*}

\subsection{Junction Depth Calculations}
\label{sec:jdepth}
Junction depth calculation for box approximation (limited-source diffusion):
\begin{equation}
x_j = 2\sqrt{Dt \ln{(N_o/N_B)}}
\end{equation}
and for a half gaussian (constant-source diffusion):
\begin{equation}
x_j = 2\sqrt{Dt}\erfc^{-1}{(N_B/N_o)}
\end{equation}
where,
\begin{align*}
D &= D_o\me{\frac{-E_A}{kT}} \,\,\text{(Diffusion coefficient)} \\ 
t &= \text{time of diffusion} \\
N_B &= \text{Background impurity concentration} \\ 
N_o &= \text{Surface concentration limited by solid solubility}
\end{align*}

Example: Calculate the junction depth after pre-diffusion and drive in. \\
According to the Process flow [2], our silicon wafer has a resistivity of about 14-16 ohm-cm. From the same process flow we know that we had a phosphorus doped, solid solubility limited constant diffusion at $1050\degree$C. Using figure 4.6 [1] we see that at a temperature of $1050\degree$C our phosphorus surface concentration $N_o \approx 10^{21}$/$\text{cm}^3$. Now using the Resistivity of our wafer and figure 4.8 [1], we see that we have an impurity concentration $N_B \approx 8\e{14}$. \\ \\
Using table 4.1 [1], we can calculate our diffusion coefficient. Note that pre-diffusion was done at $1050\degree$C for 5 minutes.
\begin{align*}
D &= D_o\me{\frac{-E_A}{kT}} = 10.5 \,\text{cm}^2/\text{sec}\,\, \exp{(\frac{-3.69*1.602\e{-19}}{1.38\e{-23}(1050 + 273)})} = 9.11\e{-14} \, \text{cm}^2/\text{sec}
\end{align*}
Now we can plug everything into our solid solubility limited box approximation equation:
\begin{align*}
x_j = 2\sqrt{Dt}\erfc^{-1}{(N_B/N_o)} = 2\sqrt{(9.11\e{-14})(300\text{sec})} \erfc^{-1}{(8\e{14}/10^{21})} \approx 365 \,\text{nm}
\end{align*}

Now for drive-in we kept the temperature the same, $1050\degree$C, but kept the wafer inside the furnace for 37 minutes (2220 seconds). Also we can no longer assume a simple box approximation, we must use a half gaussian; this means that our surface concentration is going to change with drive-in. \\\\
We will use the following equation,

\begin{equation}
N_B = (Q/\sqrt{\pi D_2t_2})\exp{(-(\frac{x_j}{2\sqrt{D_2t_2}})^2)}
\end{equation}

where Q is the does rate,
\begin{equation}
Q = 2N_o\sqrt{D_1t_1/\pi}
\end{equation}

Combining the two equations and noting that $D_1 = D_2$ because we are using the same temperature, $1050\degree$C, for both pre-diffusion and drive-in ($t_1$ is the pre-diffusion time and $t_2$ is the drive in time),

\begin{align*}
N_B = (2N_o/\pi)\sqrt{\frac{t_1}{t_2}}\exp{(-(\frac{x_j}{2\sqrt{D_2t_2}})^2)}
\end{align*}
Solving this for $x_j$ yields,
\begin{align*}
x_j = 2\sqrt{D_2t_2\ln{\big((2N_o/\pi N_B)(\sqrt{t_1/t_2})\big)}} = 2\sqrt{(9.11\e{-14})(2220)\ln{\big((2(10^{21})/\pi (8\e{14}))(\sqrt{300/2220})\big)}} \approx 1000 \,nm
\end{align*}
%----------------------------------------------------------------------------------------
%	SECTION 5
%----------------------------------------------------------------------------------------
\section{References}
\begin{enumerate}
\item Sharpe, William N., Bin Yuan, and Ranji Vaidyanathan. Measurements of Young's Modulus, Poisson's ratio, and Tensile Strength of Polysilicon. Publication no. 1. Nagoya: IEEE, 1997. Web.
\item Kim, Min. "Influence of Substrates on the Elastic Reaction of Films for the Microindentation Tests." Thin Solid Films 283 (1996): 15. Web. <http://www.sciencedirect.com/science/journal/00406090/283>.
\item Chinmulgund, M. "Effect of Ar Gas Pressure on Growth, Structure, and Mechanical Properties of Sputtered Ti, Al, TiAl, and Ti3Al Films." Thin Solid Films 270.1-2 (1995): 260-63. Web. <http://www.sciencedirect.com/science/journal/00406090/270>.
\end{enumerate}

%----------------------------------------------------------------------------------------
%	SECTION 6
%----------------------------------------------------------------------------------------

% Nothing right now

%----------------------------------------------------------------------------------------
%	BIBLIOGRAPHY
%----------------------------------------------------------------------------------------

\bibliographystyle{apalike}

\bibliography{sample}

%----------------------------------------------------------------------------------------


\end{document}

