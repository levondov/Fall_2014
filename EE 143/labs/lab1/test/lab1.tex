%%%%%%%%%%%%%%%%%%%%%%%%%%%%%%%%%%%%%%%%%
% University/School Laboratory Report
% LaTeX Template
% Version 3.1 (25/3/14)
%
% This template has been downloaded from:
% http://www.LaTeXTemplates.com
%
% Original author:
% Linux and Unix Users Group at Virginia Tech Wiki 
% (https://vtluug.org/wiki/Example_LaTeX_chem_lab_report)
%
% License:
% CC BY-NC-SA 3.0 (http://creativecommons.org/licenses/by-nc-sa/3.0/)
%
%%%%%%%%%%%%%%%%%%%%%%%%%%%%%%%%%%%%%%%%%

%----------------------------------------------------------------------------------------
%	PACKAGES AND DOCUMENT CONFIGURATIONS
%----------------------------------------------------------------------------------------

\documentclass{article}

\usepackage[version=3]{mhchem} % Package for chemical equation typesetting
\usepackage{siunitx} % Provides the \SI{}{} and \si{} command for typesetting SI units
\usepackage{graphicx} % Required for the inclusion of images
\usepackage{natbib} % Required to change bibliography style to APA
\usepackage{amsmath} % Required for some math elements 
\usepackage[utf8]{inputenc}
\usepackage{tikz,pgfplots}
\usepackage[letterpaper, margin=0.5in]{geometry}
\usepackage{float}
\usepackage{enumitem}
\usepackage{gensymb}
\usepackage[hidelinks]{hyperref}
\usepackage[all]{hypcap}
\usepackage{subfloat}
\usepackage{color}

% Roman numerials
\pagenumbering{arabic}

\setlength\parindent{0pt} % Removes all indentation from paragraphs

%\renewcommand{\labelenumi}{\alph{enumi}.} % Make numbering in the enumerate environment by letter rather than number (e.g. section 6)

%\usepackage{times} % Uncomment to use the Times New Roman font

% for some tables
\newcommand{\specialcell}[2][c]{%
  \begin{tabular}[#1]{@{}c@{}}#2\end{tabular}}
  
\providecommand{\e}[1]{\ensuremath{\times 10^{#1}}}

%%%%%%%%%%%% COLOR DEFINITIONS %%%%%%%%%%%%%

\definecolor{silicon}{RGB}{255,102,102}
\definecolor{oxide}{RGB}{145,150,110}
\definecolor{ioxide}{RGB}{175,180,135}
\definecolor{goxide}{RGB}{195,200,150}
\definecolor{poly}{RGB}{155,20,155}
\definecolor{spinglass}{RGB}{200,205,100}
\definecolor{n+}{RGB}{250,15,15}
\definecolor{aluminum}{RGB}{30,30,30}

%----------------------------------------------------------------------------------------
%	DOCUMENT INFORMATION
%----------------------------------------------------------------------------------------

%\title{Determination of the Atomic \\ Weight of Magnesium \\ CHEM 101} % Title

%\author{John \textsc{Smith}} % Author name

%\date{\today} % Date for the report

\begin{document}

%\maketitle % Insert the title, author and date

% If you wish to include an abstract, uncomment the lines below
% \begin{abstract}
% Abstract text
% \end{abstract}

%----------------------------------------------------------------------------------------
%	SECTION 1
%----------------------------------------------------------------------------------------

\begin{figure}[H]
\centering
\begin{tikzpicture}
\draw[fill=silicon] (0,1) rectangle (7,2);

\node at (6.55,1.25) {Si};
\node at (0.55,1.25) {\textbf{P}};
\node at (8.5,2.2) {};

\end{tikzpicture}
\caption{Week 1, starting wafer.}
\label{fig:1.1w1} %%%%%%%%%%%%%%%%%%%
\end{figure}

\begin{figure}[H]
\centering
\begin{tikzpicture}
\draw[fill=silicon] (0,1) rectangle (7,2);
\draw[fill=oxide] (0,2) rectangle (7,2.5);
\draw [<->] (7.1,2) -- (7.1,2.5);

\node at (6.55,1.25) {Si};
\node at (0.55,1.25) {\textbf{P}};
\node at (6.5,2.2) {Si$\text{O}_{2}$};
\node at (8,2.2) {0.52$\mu$m};

\end{tikzpicture}
\caption{Week 2, field oxidation, 0.52 $\mu$m}
\label{fig:1.1w2} %%%%%%%%%%%%%%%%%%%%%%%
\end{figure}

\begin{figure}[H]
\centering
\begin{tikzpicture}
\draw[fill=silicon] (0,1) rectangle (7,2);
\draw[fill=oxide] (0,2) -- (0,2.5) -- (0.8,2.5) parabola bend (1,2) (0.6,2) -- (0,2);
\draw[fill=oxide] (7,2) -- (7,2.5) -- (6.2,2.5) parabola bend (6,2) (6.4,2) -- (7,2);
\draw [<->] (7.1,2) -- (7.1,2.5);

\node at (6.55,1.25) {Si};
\node at (0.55,1.25) {\textbf{P}};
\node at (8,2.2) {0.52$\mu$m};
\node at (6.6,2.2) {Si$\text{O}_{2}$};

\end{tikzpicture}
\caption{Week 3, ACTV photolithography and etch}
\label{fig:1.1w3} %%%%%%%%%%%%%%%%%%
\end{figure}

\begin{figure}[H]
\centering
\begin{tikzpicture}
\draw[fill=silicon] (0,1) rectangle (7,2);
\draw[fill=oxide] (0,2) -- (0,2.5) -- (0.8,2.5) parabola bend (1,2) (0.6,2) -- (0,2);
\draw[fill=oxide] (7,2) -- (7,2.5) -- (6.2,2.5) parabola bend (6,2) (6.4,2) -- (7,2);
\draw[fill=goxide] (0.9,2) rectangle (6.1,2.077);
\draw [<->] (7.1,2) -- (7.1,2.5);
\draw [<-] (3,2.01) -- (3,2.4);

\node at (6.55,1.25) {Si};
\node at (0.55,1.25) {\textbf{P}};
\node at (8,2.2) {0.52$\mu$m};
\node at (6.6,2.2) {Si$\text{O}_{2}$};
\node at (3,2.6) {0.08$\mu$m};

\end{tikzpicture}
\caption{Week 4, gate oxidation. Gate oxide thickness is 0.08$\mu$m}
\label{fig:1.1w4} %%%%%%%%%%%%%%%%
\end{figure}

\begin{figure}[H]
\centering
\begin{tikzpicture}
\draw[fill=silicon] (0,1) rectangle (7,2);
\draw[fill=oxide] (0,2) -- (0,2.5) -- (0.8,2.5) parabola bend (1,2) (0.6,2) -- (0,2);
\draw[fill=oxide] (7,2) -- (7,2.5) -- (6.2,2.5) parabola bend (6,2) (6.4,2) -- (7,2);
\draw[fill=goxide] (0.9,2) rectangle (6.1,2.077);
% Line outline for poly
\draw[fill=poly] (0,2.5) -- (0,2.84) -- (1.24,2.84) parabola bend (1.39,2.42) (1.34,2.42) -- (5.66,2.42) parabola bend (5.61,2.42) (5.76,2.84) -- (7,2.84) -- (7,2.5) -- (6.2,2.5) parabola bend (6.05,2.077) (6.1,2.077) -- (0.9,2.077) parabola bend (0.9,2.077) (0.75,2.5) -- (0,2.5);

\draw [<->] (7.1,2) -- (7.1,2.5);
\draw [<->] (4.1,2.077) -- (4.1,2.42);

\node at (6.55,1.25) {Si};
\node at (0.55,1.25) {\textbf{P}};
\node at (8,2.2) {0.52$\mu$m};
\node at (6.6,2.2) {Si$\text{O}_{2}$};
\node at (3.5,2.2) {Poly};
\node at (4.8,2.2) {0.35$\mu$m};


\end{tikzpicture}
\caption{Week 5b, polysilicon CVD.}
\label{fig:1.1w5} %%%%%%%%%%%%%%%%%%
\end{figure}

\begin{figure}[H]
\centering
\begin{tikzpicture}
\draw[fill=silicon] (0,1) rectangle (7,2);
\draw[fill=oxide] (0,2) -- (0,2.5) -- (0.8,2.5) parabola bend (1,2) (0.6,2) -- (0,2);
\draw[fill=oxide] (7,2) -- (7,2.5) -- (6.2,2.5) parabola bend (6,2) (6.4,2) -- (7,2);
\draw[fill=goxide] (3,2) rectangle (4,2.077);
\draw[fill=poly] (3,2.077) parabola bend (3.05,2.077) (3.1,2.42) -- (3.9,2.42) parabola bend (3.95,2.077) (4,2.077) -- (3,2.077);


\draw [<->] (7.1,2) -- (7.1,2.5);
\draw [<->] (4.1,2.077) -- (4.1,2.42);

\node at (6.55,1.25) {Si};
\node at (0.55,1.25) {\textbf{P}};
\node at (8,2.2) {0.52$\mu$m};
\node at (6.6,2.2) {Si$\text{O}_{2}$};
\node at (3.5,2.2) {Poly};
\node at (4.8,2.2) {0.35$\mu$m};


\end{tikzpicture}
\caption{Week 6, polysilicon lithography and etch, Source/Drain clear}
\label{fig:1.1w6} %%%%%%%%%%%%%%%%%%%%%
\end{figure}

\begin{subfigures}
\begin{figure}[H]
\centering
\begin{tikzpicture}
\draw[fill=spinglass,opacity=0.2] (0,1) -- (7,1) -- (7,4) -- (0,4) -- (0,1); %spin on glass
\draw[fill=silicon] (0,1) rectangle (7,2);
\draw[fill=oxide] (0,2) -- (0,2.5) -- (0.8,2.5) parabola bend (1,2) (0.6,2) -- (0,2);
\draw[fill=oxide] (7,2) -- (7,2.5) -- (6.2,2.5) parabola bend (6,2) (6.4,2) -- (7,2);
\draw[fill=goxide] (3,2) rectangle (4,2.077);
\draw[fill=poly] (3,2.077) parabola bend (3.05,2.077) (3.1,2.42) -- (3.9,2.42) parabola bend (3.95,2.077) (4,2.077) -- (3,2.077);
\draw[fill=n+] (1,1.9) rectangle (3,2);
\draw[fill=n+] (4,1.9) rectangle (6,2);


\draw [<->] (7.1,2) -- (7.1,2.5);
\draw [<->] (4.1,2.077) -- (4.1,2.42);
\draw [<-] (2.2,1.75) -- (2.5,1.4);
\draw [<-] (4.8,1.75) -- (4.5,1.4);

\node at (6.55,1.25) {Si};
\node at (0.55,1.25) {\textbf{P}};
\node at (8,2.2) {0.52$\mu$m};
\node at (6.6,2.2) {Si$\text{O}_{2}$};
\node at (3.5,2.2) {Poly};
\node at (4.8,2.2) {0.35$\mu$m};
\node at (2,3.5) {Spin-On Glass};
\node at (3.5,1.3) {n+ dopant};


\end{tikzpicture}
\caption{\label{first} Week 7a, spin-on glass and source/drain predeposition}
\label{fig:1.1w7a} %%%%%%%%%%%%%%%%%%
\end{figure}
\begin{figure}[H]
\centering
\begin{tikzpicture}
\draw[fill=silicon] (0,1) rectangle (7,2);
\draw[fill=oxide] (0,2) -- (0,2.5) -- (0.8,2.5) parabola bend (1,2) (0.6,2) -- (0,2);
\draw[fill=oxide] (7,2) -- (7,2.5) -- (6.2,2.5) parabola bend (6,2) (6.4,2) -- (7,2);
\draw[fill=goxide] (3,2) rectangle (4,2.077);
\draw[fill=poly] (3,2.077) parabola bend (3.05,2.077) (3.1,2.42) -- (3.9,2.42) parabola bend (3.95,2.077) (4,2.077) -- (3,2.077);
\draw[fill=n+,rounded corners = 7.5pt] (0.9,2) --  (0.9,1.7) -- (3.1,1.7) -- (3.1,2);
\draw[fill=n+, rounded corners = 7.5pt] (3.9,2) -- (3.9,1.7) -- (6.1,1.7) -- (6.1,2);
\draw[fill=ioxide] (0.85,2.3) parabola bend (1,2) (1,2) -- (3,2) -- (3, 2.077) parabola bend (3.05,2.077)(3.1,2.42) -- (3.9,2.42) parabola bend (3.95,2.077) (4,2.077) -- (4,2) -- (6,2) parabola bend (6,2) (6.15,2.3) -- (4.2,2.3) parabola bend (4.2,2.3) (3.9,2.72) -- (3.1,2.72) parabola bend (2.8,2.3) (2.8,2.3) -- (0.85,2.3);

\draw [<->] (7.1,2) -- (7.1,2.5);
\draw [<->] (4.1,2.077) -- (4.1,2.42);
\draw[<->] (1.8,2) -- (1.8,2.3);

\node at (6.55,1.25) {Si};
\node at (2,1.85) {\textbf{n+}};
\node at (5,1.85) {\textbf{n+}};
\node at (0.55,1.25) {\textbf{P}};
\node at (8,2.2) {0.52$\mu$m};
\node at (6.6,2.2) {Si$\text{O}_{2}$};
\node at (3.5,2.2) {Poly};
\node at (4.8,2.2) {0.35$\mu$m};
\node at (1.3,2.15) {\small{Si$\text{O}_{2}$}};
\node at (2.35,2.15) {??$\mu$m};

\end{tikzpicture}
\caption{\label{second} Week 7b, drive-in and intermediate oxidation}
\label{fig:1.1w7b} %%%%%%%%%%%%%%%%
\end{figure}
\end{subfigures}

\begin{figure}[H]
\centering
\begin{tikzpicture}
\draw[fill=silicon] (0,1) rectangle (7,2);
\draw[fill=oxide] (0,2) -- (0,2.5) -- (0.8,2.5) parabola bend (1,2) (0.6,2) -- (0,2);
\draw[fill=oxide] (7,2) -- (7,2.5) -- (6.2,2.5) parabola bend (6,2) (6.4,2) -- (7,2);
\draw[fill=goxide] (3,2) rectangle (4,2.077);
\draw[fill=poly] (3,2.077) parabola bend (3.05,2.077) (3.1,2.42) -- (3.9,2.42) parabola bend (3.95,2.077) (4,2.077) -- (3,2.077);
\draw[fill=n+,rounded corners = 7.5pt] (0.9,2) --  (0.9,1.7) -- (3.1,1.7) -- (3.1,2);
\draw[fill=n+, rounded corners = 7.5pt] (3.9,2) -- (3.9,1.7) -- (6.1,1.7) -- (6.1,2);
\draw[fill=ioxide] (0.85,2.3) parabola bend (1,2) (1,2) -- (3,2) -- (3, 2.077) parabola bend (3.05,2.077)(3.1,2.42) -- (3.9,2.42) parabola bend (3.95,2.077) (4,2.077) -- (4,2) -- (6,2) parabola bend (6,2) (6.15,2.3) -- (5.2,2.3) parabola bend (5.1,2) (5.2,2) -- (4.8,2) parabola bend (4.9,2) (4.8,2.3) -- (4.2,2.3) parabola bend (4.2,2.3) (3.9,2.72) -- (3.1,2.72) parabola bend (2.8,2.3) (2.8,2.3) -- (2.2,2.3) parabola bend (2.1,2) (2.1,2) -- (1.8,2) parabola bend (1.9,2) (1.8,2.3) -- (0.85,2.3);

\draw [<->] (7.1,2) -- (7.1,2.5);

\node at (6.55,1.25) {Si};
\node at (2,1.85) {\textbf{n+}};
\node at (5,1.85) {\textbf{n+}};
\node at (0.55,1.25) {\textbf{P}};
\node at (8,2.2) {0.52$\mu$m};
\node at (6.6,2.2) {Si$\text{O}_{2}$};
\node at (3.5,2.2) {Poly};

\end{tikzpicture}
\caption{\label{second} Week 8, contact hole lithography and etch}
\label{fig:1.1w8} %%%%%%%%%%%%%%%
\end{figure}

\begin{figure}[H]
\centering
\begin{tikzpicture}
\draw[fill=silicon] (0,1) rectangle (7,2);
\draw[fill=oxide] (0,2) -- (0,2.5) -- (0.8,2.5) parabola bend (1,2) (0.6,2) -- (0,2);
\draw[fill=oxide] (7,2) -- (7,2.5) -- (6.2,2.5) parabola bend (6,2) (6.4,2) -- (7,2);
\draw[fill=goxide] (3,2) rectangle (4,2.077);
\draw[fill=poly] (3,2.077) parabola bend (3.05,2.077) (3.1,2.42) -- (3.9,2.42) parabola bend (3.95,2.077) (4,2.077) -- (3,2.077);
\draw[fill=n+,rounded corners = 7.5pt] (0.9,2) --  (0.9,1.7) -- (3.1,1.7) -- (3.1,2);
\draw[fill=n+, rounded corners = 7.5pt] (3.9,2) -- (3.9,1.7) -- (6.1,1.7) -- (6.1,2);
\draw[fill=ioxide] (0.85,2.3) parabola bend (1,2) (1,2) -- (3,2) -- (3, 2.077) parabola bend (3.05,2.077)(3.1,2.42) -- (3.9,2.42) parabola bend (3.95,2.077) (4,2.077) -- (4,2) -- (6,2) parabola bend (6,2) (6.15,2.3) -- (5.2,2.3) parabola bend (5.1,2) (5.2,2) -- (4.8,2) parabola bend (4.9,2) (4.8,2.3) -- (4.2,2.3) parabola bend (4.2,2.3) (3.9,2.72) -- (3.1,2.72) parabola bend (2.8,2.3) (2.8,2.3) -- (2.2,2.3) parabola bend (2.1,2) (2.1,2) -- (1.8,2) parabola bend (1.9,2) (1.8,2.3) -- (0.85,2.3);
\draw[fill=aluminum] (0,2.8) -- (0,2.5) -- (0.8,2.5) parabola bend (0.85,2.3) (0.85,2.3) -- (1.8,2.3) parabola bend (1.9,2) (1.8,2) -- (2.1,2) parabola bend (2.1,2) (2.2,2.3) -- (2.8,2.3) parabola bend (2.8,2.3) (3.1,2.72) -- (3.9,2.72) parabola bend (4.1,2.3) (4.2,2.3) -- (4.8,2.3) parabola bend (4.9,2) (4.9,2) -- (5.1,2) parabola bend (5.1,2) (5.2,2.3) -- (6.15,2.3) parabola bend (6.15,2.3) (6.2,2.5) -- (7,2.5) -- (7,2.8) -- (6.1,2.8) parabola bend (6.00,2.6) (6.00,2.6) -- (4.3,2.6) parabola bend (4.3,2.6) (4,3.02) -- (3,3.02) parabola bend (2.7,2.6) (2.7,2.6) -- (1,2.6) parabola bend (1,2.6) (0.9,2.8);

\draw [<->] (7.1,2) -- (7.1,2.5);
\draw [<->] (7.1,2.5) -- (7.1,2.8);

\node at (6.55,1.25) {Si};
\node at (2,1.85) {\textbf{n+}};
\node at (5,1.85) {\textbf{n+}};
\node at (0.55,1.25) {\textbf{P}};
\node at (8,2.2) {0.52$\mu$m};
\node at (6.6,2.2) {Si$\text{O}_{2}$};
\node at (3.5,2.2) {Poly};
\node at (8,2.6) {??$\mu$m};
\node at (5,2.4) {\small{\color{white} Al}};

\end{tikzpicture}
\caption{\label{second} Week 9, aluminum evaporation}
\label{fig:1.1w9} %%%%%%%%%%%%%%
\end{figure}

\begin{figure}[H]
\centering
\begin{tikzpicture}
\draw[fill=silicon] (0,1) rectangle (7,2);
\draw[fill=oxide] (0,2) -- (0,2.5) -- (0.8,2.5) parabola bend (1,2) (0.6,2) -- (0,2);
\draw[fill=oxide] (7,2) -- (7,2.5) -- (6.2,2.5) parabola bend (6,2) (6.4,2) -- (7,2);
\draw[fill=goxide] (3,2) rectangle (4,2.077);
\draw[fill=poly] (3,2.077) parabola bend (3.05,2.077) (3.1,2.42) -- (3.9,2.42) parabola bend (3.95,2.077) (4,2.077) -- (3,2.077);
\draw[fill=n+,rounded corners = 7.5pt] (0.9,2) --  (0.9,1.7) -- (3.1,1.7) -- (3.1,2);
\draw[fill=n+, rounded corners = 7.5pt] (3.9,2) -- (3.9,1.7) -- (6.1,1.7) -- (6.1,2);
\draw[fill=ioxide] (0.85,2.3) parabola bend (1,2) (1,2) -- (3,2) -- (3, 2.077) parabola bend (3.05,2.077)(3.1,2.42) -- (3.9,2.42) parabola bend (3.95,2.077) (4,2.077) -- (4,2) -- (6,2) parabola bend (6,2) (6.15,2.3) -- (5.2,2.3) parabola bend (5.1,2) (5.2,2) -- (4.8,2) parabola bend (4.9,2) (4.8,2.3) -- (4.2,2.3) parabola bend (4.2,2.3) (3.9,2.72) -- (3.1,2.72) parabola bend (2.8,2.3) (2.8,2.3) -- (2.2,2.3) parabola bend (2.1,2) (2.1,2) -- (1.8,2) parabola bend (1.9,2) (1.8,2.3) -- (0.85,2.3);
\draw [fill=aluminum] (1.5,2.3) -- (1.8,2.3) parabola bend (1.9,2) (1.8,2) -- (2.1,2) parabola bend (2.1,2) (2.2,2.3) -- (2.5,2.3) parabola bend (2.5,2.3) (2.4,2.6) -- (1.6,2.6) parabola bend (1.5,2.3) (1.5,2.3);
\draw [fill=aluminum] (4.5,2.3) -- (4.8,2.3) parabola bend (4.9,2) (4.9,2) -- (5.1,2) parabola bend (5.1,2) (5.2,2.3) -- (5.5,2.3) parabola bend (5.5,2.3) (5.4,2.6) -- (4.6,2.6) parabola bend (4.5,2.3) (4.5,2.3); 

\draw [<->] (7.1,2) -- (7.1,2.5);

\node at (6.55,1.25) {Si};
\node at (2,1.85) {\textbf{n+}};
\node at (5,1.85) {\textbf{n+}};
\node at (0.55,1.25) {\textbf{P}};
\node at (8,2.2) {0.52$\mu$m};
\node at (6.6,2.2) {Si$\text{O}_{2}$};
\node at (3.5,2.2) {Poly};
\node at (1.3,2.15) {\small{Si$\text{O}_{2}$}};
\node at (5,2.4) {\small{\color{white} Al}};


\end{tikzpicture}
\caption{\label{second} Week 10, metal lithography and etch}
\label{fig:1.1w10} %%%%%%%%%%%%%%%%
\end{figure}
%----------------------------------------------------------------------------------------
%	SECTION 5
%----------------------------------------------------------------------------------------


%----------------------------------------------------------------------------------------
%	SECTION 6
%----------------------------------------------------------------------------------------

% Nothing right now

%----------------------------------------------------------------------------------------
%	BIBLIOGRAPHY
%----------------------------------------------------------------------------------------

\bibliographystyle{apalike}

\bibliography{sample}

%----------------------------------------------------------------------------------------


\end{document}

