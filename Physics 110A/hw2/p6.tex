\documentclass[12pt]{article}

\usepackage{listings}
\usepackage{color}
\usepackage[letterpaper, portrait, margin=1in]{geometry}

\definecolor{dkgreen}{rgb}{0,0.6,0}
\definecolor{gray}{rgb}{0.5,0.5,0.5}
\definecolor{mauve}{rgb}{0.58,0,0.82}

\lstset{frame=tb,
  language=Python,
  aboveskip=3mm,
  belowskip=3mm,
  showstringspaces=false,
  columns=flexible,
  basicstyle={\small\ttfamily},
  numbers=none,
  numberstyle=\tiny\color{gray},
  keywordstyle=\color{blue},
  commentstyle=\color{dkgreen},
  stringstyle=\color{mauve},
  breaklines=true,
  breakatwhitespace=true
  tabsize=3
}

\begin{document}
\begin{lstlisting}
import numpy as np

chargeX = [1,1,1,1,-1,-1,-1,-1]
chargeY = [1,1,-1,-1,-1,-1,1,1]
chargeZ = [1,-1,1,-1,-1,1,-1,1]

point1 = [0.01,0,0]
point2 = [0.01,0.01,0]
point3 = [0.01,0.01,0.01]
point4 = [0,0,0]

# method to calculate total E field in square given a source point
def total_E_field(field_coord):
	#field_coord should be an array of form (x,y,z)
	Ex = 0
	Ey = 0
	Ez = 0
	
	for i in range(0,8):
		charge = [chargeX[i],chargeY[i],chargeZ[i]]
		
		# calculate 3d & 2d distance and Exyz field
		distance2d = np.linalg.norm(np.array(field_coord[0:2]) - np.array(charge[0:2]))
		distance3d = np.linalg.norm(np.array(field_coord) - np.array(charge))
		Exyz = 1./(distance3d*distance3d)
		
		# calculate z component of E field
		Ez += (np.abs(field_coord[2]-charge[2])/distance3d)*Exyz
		
		# calculate horizontal plane field
		Exy = (distance2d/distance3d)*Exyz

		# calculate theta2 to help find Ex,Ey
		theta2 = np.arctan(np.abs(field_coord[0] - charge[0])/np.abs(field_coord[1] - charge[1]))
		
		# calculate the x and y component of the E field
		Ey += np.cos(theta2)*Exy
		Ex += np.sin(theta2)*Exy
		
	return Ex,Ey,Ez

# call method and print out results
Ex1,Ey1,Ez1 = total_E_field(point1)
Ex2,Ey2,Ez2 = total_E_field(point2)
Ex3,Ey3,Ez3 = total_E_field(point3)
Exc,Eyc,Ezc = total_E_field(point4)

print('E1 = ',Ex1,Ey1,Ez1)
print('E2 = ',Ex2,Ey2,Ez2)
print('E3 = ',Ex3,Ey3,Ez3)
print('E4 = ',Exc,Eyc,Ezc)
\end{lstlisting}
Output after running, \\ \\
('E1 = ', 1.5394980763658117, 1.5396520335862838, 1.5396520335862833) \\
('E2 = ', 1.539549396388431, 1.5395493963884312, 1.5397033707155778) \\
('E3 = ', 1.539600737801047, 1.539600737801047, 1.539600737801047) \\
('E4 = ', 1.5396007178390023, 1.5396007178390025, 1.5396007178390023)

\end{document}

