% --------------------------------------------------------------
% This is all preamble stuff that you don't have to worry about.
% Head down to where it says "Start here"
% --------------------------------------------------------------
 
\documentclass[12pt]{article}
 
\usepackage[margin=1in]{geometry} 
\usepackage{amsmath,amsthm,amssymb}
\usepackage{gensymb}
\usepackage{graphicx}
 \usepackage{tikz,pgfplots}
\usepackage{float}
\usepackage{enumitem}
\usepackage[utf8]{inputenc}


\newcommand{\N}{\mathbb{N}}
\newcommand{\Z}{\mathbb{Z}}

\newcommand{\slantedgrid}[4]{%
   \pgfmathtruncatemacro{\result}{#1+#3}
   \foreach \x in {#1,...,\result} \draw (\x,#2) -- ++(#4,#4);%
   \pgfmathtruncatemacro{\result}{#2+#4}
   \foreach \y in {#2,...,\result} \draw (#1+\y-#2,\y) -- ++(#3,0);%
 }
 \newcommand{\specialcell}[2][c]{%
  \begin{tabular}[#1]{@{}c@{}}#2\end{tabular}}
  
\DeclareMathOperator\erf{erf}
 
\newenvironment{theorem}[2][Theorem]{\begin{trivlist}
\item[\hskip \labelsep {\bfseries #1}\hskip \labelsep {\bfseries #2.}]}{\end{trivlist}}
\newenvironment{lemma}[2][Lemma]{\begin{trivlist}
\item[\hskip \labelsep {\bfseries #1}\hskip \labelsep {\bfseries #2.}]}{\end{trivlist}}
\newenvironment{exercise}[2][Exercise]{\begin{trivlist}
\item[\hskip \labelsep {\bfseries #1}\hskip \labelsep {\bfseries #2.}]}{\end{trivlist}}
\newenvironment{problem}[2][Problem]{\begin{trivlist}
\item[\hskip \labelsep {\bfseries #1}\hskip \labelsep {\bfseries #2.}]}{\end{trivlist}}
\newenvironment{question}[2][Question]{\begin{trivlist}
\item[\hskip \labelsep {\bfseries #1}\hskip \labelsep {\bfseries #2.}]}{\end{trivlist}}
\newenvironment{corollary}[2][Corollary]{\begin{trivlist}
\item[\hskip \labelsep {\bfseries #1}\hskip \labelsep {\bfseries #2.}]}{\end{trivlist}}
 
\begin{document}
\providecommand{\e}[1]{\ensuremath{\times 10^{#1}}}
\providecommand{\ex}[1]{\ensuremath{10^{#1}}}
% --------------------------------------------------------------
%                         Start here
% --------------------------------------------------------------
 
\title{HW 10}
\author{Levon Dovlatyan \\ SI: 24451582\\ E45} 
\date{Dec 5, 2014}
\maketitle
 
\begin{problem}{13.25} 
What is the power dissipation ($I^2$R) in the filament of Problem 13.24?.
\end{problem}
\begin{problem}{Ref 13.24}
A tungsten lightbulb filament is 9 mm long and 100 $\mu$m in diameter.  What is the current in the filament when operating at 1,000$\degree$C with a line voltage of 110 V?  
\end{problem}

Using eqn 13.9 in the textbook[1], we first solve for resistivity where $T_{th}$ is room temperature and $\rho_{th}$ is room resistivity. $\alpha$ is a constant we grab from table 13.2 in the book.
\begin{align*}
\rho = \rho_{th}[1 + \alpha(T-T_{th})] = (55.1\e{-9})(1 + 0.0045(1000-20)) = 2.98\e{-7}\,\Omega m
\end{align*}

Now we solve for Resistance:
\begin{align*}
R = \frac{\rho l}{A} = \frac{2.98\e{-7} * 9\e{-3}}{\pi*(100\e{-6}/2)^2} = 0.341 \Omega
\end{align*}

Now we solve for Power emission,
\begin{align*}
P = I^2 R = V^2 / R = (110 V)^2 / (0.341) = 35.5\e{3} \text{W}
\end{align*}

\begin{problem}{13.35}
An alternate definition of polarization (introduced in Examples 13.11 and 13.12) is
\begin{equation}
P = (\kappa–1)\epsilon_0 E,
\end{equation}
where $\kappa$, $\epsilon_0$, and $E$ were defined relative to Equations 13.11 and 13.13.  Calculate the polarization for 99.9\% Al2O3 under a field strength of 5 kV/mm.  (You might note the magnitude of your answer in comparison to the inherent polarization of tetragonal BaTiO3 in Example 13.12).
\end{problem}

From table 13.4 [1], $\kappa = 10.1$ and $\epsilon_0 = 8.85\e{-12} C/V m$. Also note that 5 kV/mm is the same as $5*1000*1000 V/m$.
\begin{align*}
P = (\kappa–1)\epsilon_0 E = (10.1-1)*8.85\e{-12}*5*1000*1000 V/m = 4.03\e{-4} C/\text{m}^2
\end{align*}

\begin{problem}{13.45}
What fraction of the conductivity at room temperature for \textbf{(a)} germanium and \textbf{(b)} CdS is contributed by (i) electrons and (ii) electron holes?
\end{problem}
Using eqn 13.14[1] in the textbook we cal solve this. Note that:
\begin{align*}
\sigma_{\text{total}} =& nq(\mu_e + \mu_h) \\
\sigma_e =& nq\mu_e \\
\sigma_h =& nq\mu_h \\
\end{align*}
Now to find the percent conductivity of electrons and holes we simply divide the conductivity due to electrons/holes by the total conductivity.
\begin{align*}
\frac{\sigma_e}{\sigma_{\text{total}}} =& \frac{nq\mu_e}{nq(\mu_e + \mu_h)} = \frac{\mu_e}{\mu_e + \mu_h} \\ \\
\frac{\sigma_h}{\sigma_{\text{total}}} =& \frac{nq\mu_h}{nq(\mu_e + \mu_h)} = \frac{\mu_h}{\mu_e + \mu_h}
\end{align*}
\textbf{(a)} Using table 13.5[1] we find that for Germanium we have:
\begin{align*}
\frac{\sigma_e}{\sigma_{\text{total}}} =& \frac{0.364}{0.364 + 0.190} = 0.66 \approx 66\% \\ \\
\frac{\sigma_h}{\sigma_{\text{total}}} =& \frac{0.190}{0.364 + 0.190} = 0.34 \approx 34\%
\end{align*}
\textbf{(b)} For CdS we have:
\begin{align*}
\frac{\sigma_e}{\sigma_{\text{total}}} =& \frac{0.034}{0.034 + 0.0018} = 0.95 \approx 95\% \\ \\
\frac{\sigma_h}{\sigma_{\text{total}}} =& \frac{0.0018}{0.034 + 0.0018} = 0..050 \approx 5.0\%
\end{align*}

\begin{problem}{15.21}
 The densities for three iron oxides are FeO (5.70 Mg/$m^3$), $\text{Fe}_3\text{O}_4$ (5.18 Mg/$m^3$), and $\text{Fe}_2\text{O}_3$ (5.24 Mg/$m^3$).  Calculate the Pilling-Bedworth ratio for iron relative to each type of oxide, and comment of the implications for the formation of a protective coating.
\end{problem}

From the textbook[1] we see that iron has an amu of 55.85 and a density of 7.87 g/$\text{cm}^3$. Also note for eqn 14.13[1] in the textbook that M = amu of the oxide, m = amu of iron, D = density of oxide, d = density of iron, and a = amount of iron in oxide.

\textbf{(a)} For FeO we have a amu of $55.85 + 16.0$ and a density of 5.70 g/$\text{cm}^3$.
\begin{align*}
R = \frac{Md}{amD} = \frac{(55.85 + 16)(7.87)}{1(55.85)(5.70)} = 1.78
\end{align*}
Since the R value here is between 1 and 2, this means the oxide will allow for a coating layer to form.

\textbf{(b)} For $\text{Fe}_3\text{O}_4$ we have a amu of $3(55.85) + 4(16.0)$ and a density of 5.18 g/$\text{cm}^3$. The formula also contains 3 Fe atoms so a = 3 here.
\begin{align*}
R = \frac{Md}{amD} = \frac{(3(55.85) + 4(16))(7.87)}{3(55.85)(5.18)} = 2.10
\end{align*}
Since the R value here is greater than 2 it will probably not form a protective coating because of the compressive stresses due to the oxide which will cause it to flake and fall off.

\textbf{(c)} For $\text{Fe}_2\text{O}_3$ we have a amu of $2(55.85) + 3(16.0)$ and a density of 5.24 g/$\text{cm}^3$. The formula also contains 2 Fe atoms so a = 2 here.
\begin{align*}
R = \frac{Md}{amD} = \frac{(2(55.85) + 3(16))(7.87)}{2(55.85)(5.24)} = 2.15
\end{align*}
Similarly, since the R value here is greater than 2 it will probably not form a protective coating because of the compressive stresses due to the oxide which will cause it to flake and fall off.


\begin{problem}{15.39}
The maximum corrosion current density in a galvanized steel sheet used in the design of the new engineering laboratories on campus is found to be 5 mA/$m^2$.  What thickness of the zinc layer is necessary to ensure at least (a) 1 year and (b) 5 years of rust resistance?
\end{problem}

The solution to this problem is essentially playing with units. We first note the density of zinc is 7.13\e{6}g/$m^3$. Also the amu of zinc is 65.38g and that zinc has 2 electrons that it gives away when corrosion occurs.

\begin{align*}
\frac{5\e{-3}A}{m^2} = \frac{5\e{-3}C}{\text{s}*m^2}\frac{1e}{1.602\e{-19}C}\frac{1\,\text{zinc atom}}{2 e}\frac{65.38g}{6.023\e{23} \,\text{atoms}} = 1.69\e{-6} \frac{g}{\text{s}*m^2}
\end{align*}
Now this is the amount of of zinc leaving the surface of the metal per a second per a meter squared area. Let's set this equal to the density of zinc and solve for units.
\begin{align*}
1.69\e{-6} \frac{g}{\text{s}*m^2} = (7.13\e{6}\,\frac{g}{m^3})(x\,\, \text{meters})
\end{align*}
Now we need to multiply by some number x in order to match our units. This x also happens to be the thickness of zinc leaving the surface per a second.
\begin{align*}
x = \frac{1.69\e{-6}\frac{g}{\text{s}*m^2}}{7.13\e{6}\,\frac{g}{m^3}} = 2.37\e{-13}\,m/s
\end{align*}
Now the problem asks us for the amount that leaves in a year and 5 years. We simply multiple and convert our units to years.
\begin{align*}
\text{\textbf{(a)}} =& 2.37\e{-13}\,m/s\,*\frac{3600 s}{1 \text{hr}}\frac{24 \text{hr}}{1 \text{day}}\frac{365\text{day}}{\text{1year}} = 7.47\mu \text{m} / \text{year} \\ \\
\text{\textbf{(b)}} =& 2.37\e{-13}\,m/s\,*\frac{3600 s}{1 \text{hr}}\frac{24 \text{hr}}{1 \text{day}}\frac{1825\text{day}}{\text{5year}} = 37.4 \mu \text{m} / \text{5 year}
\end{align*}


\section{References}
\begin{enumerate}
\item James F. Shackelford, Introduction to Materials Science for Engineers, Seventh Edition, Pearson Higher Education, Inc., Upper Saddle River, New Jersey (2009).
\end{enumerate}





% --------------------------------------------------------------
%     You don't have to mess with anything below this line.
% --------------------------------------------------------------
 
\end{document}
